\documentclass[12pt,onecolumn]{IEEEtran}
\usepackage{cite}
\usepackage{float}
\usepackage{hyperref}
\usepackage{graphicx}
\usepackage{caption}
\graphicspath{{./images/}}

\newcommand{\im}[3]{\begin{figure}[H]\begin{center}\includegraphics[scale=#1]{#2}\caption{#3}\captionsetup{justification=centering}\end{center}\end{figure}}

\title{The Thermodynamics of a Ramjet}
\author{%
  \IEEEauthorblockN{%
    \parbox{\linewidth}{\centering
	  Drake, G.\IEEEauthorrefmark{1}    
      Honeysett, R.\IEEEauthorrefmark{2},
      Johnston, C.\IEEEauthorrefmark{3},
      Khela, M.\IEEEauthorrefmark{4}%
      }%
      }
      \IEEEauthorblockA{%
      University of Edinburgh\\
      Email:\IEEEauthorrefmark{1}s1792587@ed.ac.uk
      \IEEEauthorrefmark{2}s1711116@ed.ac.uk,
      \IEEEauthorrefmark{3}s1711493@ed.ac.uk,
      \IEEEauthorrefmark{4}s1709582@ed.ac.uk%
      }%
      }
\date{}

\begin{document}

\maketitle
\im{0.7}{A_Real_Ramjet}{A NACA engineer cleaning a ramjet circa. 1950 \cite{nasa}}
\section{Introduction}
A ramjet is an airbreathing engine which uses forward motion to compress air against a static conical compressor. As such, it can only generate thrust when already in motion. A typical ramjet operates from speeds of Mach 3 to Mach 6. This report will analyse a ramjet through the lens of thermodynamics, using idealised Brayton cycles to dissect the sections of the jet and the state variables in each section.
\section{The Ramjet}
\paragraph{Inlet}
test
\paragraph{Combustion Chamber}
\paragraph{Nuzzles your Ramjet}
UwU OwO Owo vWv
\section{Efficiency}
The efficiency of the ramjet is the ratio of the propulsive power to the fuel power.\cite{greitzer_spakovsky_waitz}\\
\begin{center}
$ \eta = \frac{Tv}{\dot{m}_f h}. $
\end{center}
To derive a working expression for the efficiency, consider the thermal and propulsion aspects of the efficiency individually.\\
\begin{center}
$ \eta = \eta_{thermal} \eta_{propulsive}$
\end{center}
where\\
\begin{center}
$
\eta_{thermal}=\frac{\Delta E_K}{\dot{m}_{fuel}h}
$
$
\eta_{propulsive}=\frac{Tv}{\Delta E_K}
$
\vspace{1mm}
where\\
\vspace{1mm}
$
E_K = $ Kinetic energy (J)\\
\vspace{1mm}
$\dot{m}_{fuel} = $ Rate of fuel burned $(kgs^{-1})$\\
\vspace{1mm}
$h = $ Fuel energy per unit mass $(Jkg^{-1})$\\
\vspace{1mm}
$T = $ Thrust (N)\\
\vspace{1mm}
$v = $ Velocity of the air entering the ramjet $(ms^{-1})$\\
\end{center}
\section{Conclusion}

\bibliography{references}
\bibliographystyle{IEEEtran}

\end{document}