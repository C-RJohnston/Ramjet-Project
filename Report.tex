\documentclass[12pt,onecolumn]{IEEEtran}
\usepackage{cite}
\usepackage{float}
\usepackage{hyperref}
\usepackage{graphicx}
\usepackage{caption}
\graphicspath{{./images/}}

\newcommand{\im}[3]{\begin{figure}[H]\begin{center}\includegraphics[scale=#1]{#2}\caption{#3}\captionsetup{justification=centering}\end{center}\end{figure}}

\title{The Thermodynamics of a Ramjet}
\author{%
  \IEEEauthorblockN{%
    \parbox{\linewidth}{\centering
	  Drake, G.\IEEEauthorrefmark{1}    
      Honeysett, R.\IEEEauthorrefmark{2},
      Johnston, C.\IEEEauthorrefmark{3},
      Khela, M.\IEEEauthorrefmark{4}%
      }%
      }
      \IEEEauthorblockA{%
      University of Edinburgh\\
      Email:\IEEEauthorrefmark{1}s1792587@ed.ac.uk
      \IEEEauthorrefmark{2}s1711116@ed.ac.uk,
      \IEEEauthorrefmark{3}s1711493@ed.ac.uk,
      \IEEEauthorrefmark{4}s1709582@ed.ac.uk%
      }%
      }
\date{}

\begin{document}

\maketitle
\im{0.7}{A_Real_Ramjet}{A NACA engineer cleaning a Ramjet circa. 1950 \cite{nasa}}
\section{Introduction}
A Ramjet is an airbreathing engine that compresses air through a choke point without using active compression. As such, it requires to already be moving at speed to function. A typical ramjet operates from speeds of Mach 3 to Mach 6. This report will analyse a Ramjet through the lens of thermodynamics, using idealised Brayton cycles to dissect the sections of the jet and the state variables at each section.
\section{The Brayton Cycle and The Ramjet}
\im{0.6}{Brayton-cycle}{An idealised Brayton Cycle \cite{brayton}}
\subsection{Inlet and Compressor}
\subsection{Combustor}
\subsection{Turbine and Nozzle}
\subsection{Heat Rejection to Atmosphere}
\section{Conclusion}

\bibliography{references}
\bibliographystyle{IEEEtran}

\end{document}
